\documentclass[a4paper, 11pt]{article}
\usepackage[utf8]{inputenc}
\usepackage[left=2cm,text={17cm, 24cm},top=3cm]{geometry}
\usepackage{times}
\usepackage[czech]{babel}
\usepackage{etoolbox}
\apptocmd{\sloppy}{\hbadness 10000\relax}{}{}



\begin{document}
 \begin{titlepage} 
    \begin{center}
        \thispagestyle{empty}
        \Huge \textsc{Vysoké učení technické v Brně}\\
        \huge \textsc{Fakulta informačních technologií}

        \vspace{\stretch{0.382}}

        \LARGE Typografie a publikování – 4. projekt

        \Huge OLED
        \vspace{\stretch{0.618}}
        \end{center}
    {\Large \today  \hfill Tomáš Tomala (xtomal02)}
 \end{titlepage}

\section*{Úvod} 
I když jsme se o organických \uv{ledkách} mohli dočíst v odborných časopisech již v roce 2004 \cite{MaximumPC}, jejich využití na trhu jsme pořádně mohli spatřit až v posledních letech. Jsou to polovodičové součástky o velikost 100 nm a~500 nm, které vyzařují světlo \cite{BartosikThesis}.

Průmysl s nejnovějším typem diod se ale teprve rozvíjí a je v rané etapě. Trh je v tento moment monopolizován firmami Samsung a LG \cite{MarkShar}.

\section{Využití}

Využití se našlo hlavně v odvětví mobilních telefonů, televizí a monitorů. Téměř každá vlajková loď výrobce chytrých telefonů je v této době osazena displejem s OLED technologií. 

OLED displeje díky své možnosti ohnutí jsou perfektním produktem pro požadavky moderního telefonu s malými rámečky \cite{OLEDINFO}. Také u televize můžeme narazit na OLED displej. Díky této technologie nyní máme prohnuté obrazovky monitorů a televizí, které nabízí dechberoucí divácký zážitek.
\section{Konstrukce}
V diodě jsou 2 základní vrstvy: Anoda a katoda \cite{HruskaThesis}. Mezi tyto vrstvy jsou umístněny organické vrstvy, které pomáhají přepravovat \uv{díry} od anody a elektrony od katody. Na diodu je poté přidána podpůrná vrstva, někdy nazývána substrát.

Známe 2 nejdůležitější typy konstrukcí OLED displejů: PMOLED a AMOLED \cite{FundandAppls}.
\subsection{PMOLED}
Každý řádek displeje je ovládán jako celek. Výsledek je jednoduché řízení displeje a malá výrobní cena.\\
Nevýhoda tohoto typu je neefektivita oproti AMOLED typu a limit rozlišení displeje \cite{PCMag}.
\subsection{AMOLED}
Displeje AMOLED jsou také řízeny po řádku. Narozdíl od PMOLED ale AMOLED displej obsahuje na každý řádek kapacitor, který se stará o stavy jednotlivých pixelů \cite{DispandLight}. Výrobní cena a efektivita je vyšší. Limit na rozlišení displeje je odstraněn.
\section{Charakteristika}
\subsection{Výhody}
\begin{itemize}
        \setlength\itemsep{0.05em}
        \item perfektní pozorovací úhly \cite{PACOLED},
        \item velmi energeticky úsporné \cite{ElekCasop},
        \item možnost perfektní černé. U typu AMOLED se pixel dokáže při černé úplně vypnout,
        \item ohebnost.
    \end{itemize}
\subsection{Nevýhody}
\begin{itemize}
        \setlength\itemsep{0.05em}
        \item náročnější a dražší výrobní proces,
        \item modré diody mají oproti zelené a červené násobně menší žívotnost \cite{MaximumPC}.
    \end{itemize}
  
 \newpage
 \renewcommand{\refname}{Literatura}
 \bibliography{lit}
 \bibliographystyle{czechiso}


\end{document}
